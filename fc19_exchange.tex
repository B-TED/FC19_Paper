%
%Paper for FC 2019
%Title: How to encourage cryptocurrency exchange to secure their system
%Authors: Shin'ichiro Matsuo, Yuta Takanashi, Pindar Wong, Eric Burger
%

\documentclass[english]{llncs}
\usepackage[T1]{fontenc}
\usepackage[latin9]{inputenc}
\usepackage[dvipdfmx]{graphicx}

\usepackage{babel}

%\title{What is Cryptocurrency Exchange? - the reality and how to secure it}
\title{Reality of less trust than expected to trustless financial system}

\titlerunning{Reality of less trust than expected to trustless financial system}
\author{
Yuta Takanashi\inst{1,2}\and Shin'ichiro Matsuo \inst{1} \and Pindar Wong\inst{3} \and Eric Burger\inst{1}}
\institute{Georgetown University
\and Financial Services Agency
\and VeriFi Ltd.}
\date{\today} 

\begin{document}
\maketitle

\begin{abstract}
After the original publication of the Bitcoin paper~\cite{N08}, cryptocurrency exchange was born to connect the cryptocurrency world 
to fiat currency world. Although most persons believe that the cryptocurrency exchange plays the simple role of exchange, 
several security incidents raise a big question on the reality of the cryptocurrency exchange. That is, what kinds of operations
are conducted, whats are the informational asset to be protected, and what kinds of security postures are required.

In this paper, we report the results of investigation of 16 registered and 16 quasi-registered cryptocurrency
exchanges by Japanese regulators, then analyze the reality of functionalities, implementation and operations of real cryptocurrency exchanges.
Then, we conduct analysis from the information security management system (ISMS) standard point of views, to clarify
the required actions to secure the implementation and operation of cryptocurrency exchanges. They includes analysis on
blockchain protocol, cryptographic key management, system security, and operation. We shows the current activities in IETF and 
ISO to have a common document on security of cryptocurrency exchange.

{\bf Keywords:} Cryptocurrency exchange, Information security management, Standards
\end{abstract}

\section{Introduction}          
\subsection{Background}
Eliminating the trusted party in realizing network based services is one of 
the biggest dreams in financial cryptographic research, hence this is the main subject in this world. 
The main reason why we seek
to eliminate the trusted party is, it is too difficult to realize expected trusted party. Such difficulties are caused by operator's mistakes, malicious activities and 
collusion with other parties. Many cryptographic techniques like secret sharing scheme, threshold cryptography and multi party computation
protocol are well studied to realize many network based services without trusted parties. 
The sentence, ``An electronic payment system based on cryptographic proof instead of trust, allowing any two willing parties to transact directly with each other without the need for a trusted third party.'' is the explanation of Bitcoin - one of the most attractive cryptographic protocols - described in 
the original paper~\cite{N08}. Bitcoin is the one of the most excellent cryptographic protocols which 
claimed to realize payment scheme among cryptographic protocols which try to eliminate the trusted party.

In spite of this attractive claim of Bitcoin, there is a fundamental assumption in the claim. That is, it holds only when the payment is conducted by 
Bitcoin, and no exchange to any other payment methods exits. 




\subsection{Contributions}





\section{Investigation and audit to 32 cryptocurrency exchanges}
\subsection{Security incidents and their history}
\subsection{Perspectives of investigations}
\subsection{Results of investigation and audit}



\section{Analysis of the reality of ``Cryptocurrency Exchanges''}
\subsection{How do cryptocurrency exchanges introduce themselves and general persons recognize it}
\subsection{Functionalities which real cryptocurrency exchanges have}
\subsection{Shortage of security consideration}






\section{Reconsidering governance and security management}
\subsection{Threat modeling and security requirements}
\subsection{Analysis based on governance and security management standard}




\section{Directions and action items to secure cryptocurrency exchanges}
\subsection{Required technologies}
\subsection{Required operations}
\subsection{Standardization}




\section{Conclusion}

\begin{thebibliography}{21}
\bibitem{N08} Satoshi Nakamoto, ``Bitcoin: A Peer-to-Peer Electronic Cash System,'' \textit{https://bitcoin.org/bitcoin.pdf}.

\bibitem{IETF18}
M. Sato, M. Shimaoka, and H. Nakajima, ``  General Security Considerations for Crypto Assets Custodians,''
 \textit{https://tools.ietf.org/html/draft-vcgtf-crypto-assets-security-considerations-02}

\bibitem{ISOTR23576}
S. Matsuo and A. L. Castro (ed.), ``ISO TR 23576: Blockchain and distributed ledger technologies -- Security of digital asset custodians,''  under development,
\textit{https://www.iso.org/standard/76072.html}


\end{thebibliography}

\end{document}
